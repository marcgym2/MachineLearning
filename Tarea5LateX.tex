\documentclass{article}
\usepackage[utf8]{inputenc}
\usepackage{graphicx}

\title{Tarea 5}
\author{Marco Antonio Obregón Flores}
\date{18 de febrero de 2023}

\begin{document}

\begin{titlepage}
    \centering
    {\scshape\large Facultad de Ciencias Físico Matemáticas\par}
    {\scshape\large Maestría en Ciencia de Datos\par}
    \vspace{1cm}
    {\huge\bfseries Density-based spatial clustering of applications with noise\par}
    \vspace{2cm}
    {\Large\itshape Marco Antonio Obregón Flores\par}
    \vfill
    {\large Profesor:\par}
    {\Large\itshape José Alberto Benavides Vazquez\par}
    \vspace{1cm}
    {\large 18 de febrero de 2023\par}
\end{titlepage}

\subsection{NearestNeighbors}
\setlength{\parskip}{15pt}

El código utiliza la clase NearestNeighbors de scikit-learn para encontrar los k vecinos más cercanos para cada punto en el conjunto de datos. Luego, ordena las distancias de forma ascendente y calcula la curva de k-distancia. La curva de k-distancia muestra la distancia media al k-ésimo vecino más cercano para cada punto, ordenados de forma creciente.

El código busca la posición del "codo" en la curva de k-distancia, que corresponde a un punto donde el cambio en la distancia media comienza a disminuir drásticamente. Este punto indica el número de $min samples$
apropiado para los datos.

\begin{figure}[h]
\centering
\includegraphics[width=0.7\textwidth]{reachability_plot.png}
\caption{Descripción de la imagen}
\label{fig:reachability_plot}
\end{figure}

\subsection{DBSCAN}

Usando los resultados de NearestNeighbors, se aplicó el algoritmo DBSCAN con un valor de $\epsilon = 0.15$ y un valor de $min\_samples = 1$. El resultado fue la siguiente agrupación:

\begin{itemize}
\item Número de grupos encontrados: 26
\item Número de puntos considerados como ruido: 0
\end{itemize}

\begin{figure}[h]
\centering
\includegraphics[width=0.7\textwidth]{DBSCAN.png}
\caption{puntos centrales y puntos frontera}
\label{fig:reachability_plot}
\end{figure}

De lo anterior, podemos observar que no todos los grupos encontrados son necesariamente buenos o útiles. En algunos casos, los grupos pueden ser un artefacto del ruido o de las fluctuaciones aleatorias en los datos, o pueden ser demasiado pequeños o poco significativos para ser útiles. 

Es importante mencionar que al reducir la cantidad de datos y la dimensionalidad de los mismos, con el proposito de evitar el consumo excesivo de memoria, puede perderse información valiosa.

\end{document}


